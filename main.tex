\documentclass[11pt]{report}

% encoding
% ------------------------------
\usepackage[T1]{fontenc}
\usepackage[utf8]{inputenc}

% Pacotes para Portugues
% ------------------------------
\usepackage[portuguese]{babel}

% Regra para hifens
% ------------------------------
\usepackage{hyphenat}
\hyphenation{mate-mática recu-perar}

% Outros pacotes (utilitarios)
% ------------------------------
\usepackage{hyperref}

\usepackage{float}

\usepackage{graphicx}
\graphicspath{ {./Imagens/} }

\usepackage{algorithm}
\usepackage{algpseudocode}

\usepackage{ragged2e}

\usepackage{amsthm}

% Definições 
% -----------------------------
\theoremstyle{definition}
\newtheorem{definition}{Definição}


\begin{document}

\title{\includegraphics[width=0.5\textwidth]{minho.png}~\\[1cm] O Sistema Merkle-Hellman Knapsack}

\author{Bernardo Rodrigues\\ \texttt{a79008@alunos.uminho.pt}\\ \and César Silva\\ \texttt{a77518@alunos.uminho.pt}\\ \and Maria Franisca Tavares?\\ \texttt{a99999@alunos.uminho.pt}\\}

\date{Universidade do Minho --- \today}

\maketitle

\tableofcontents

\begin{abstract}

	Coisas nunca antes ditas

\end{abstract}

\chapter{Introdução}

Este trabalho foi desenvolvido no ambito da Unidade Curricular de \texttt{Teoria de Números Computacional}. De entre as escolhas possiveis, foi escolhido estudar o sistema \textit{Merkle-Hellman Knapsack}. \\
Este foi um dos pioneiros da criptografia de chave pública, inventado por \textbf{Ralph Merkle} e por \textbf{Martin Hellman} em 1978.
A ideia por detrás deste sistema é mais simples do que a de sistemas como o \textit{RSA}, assentando no problema  (tendo já sido quebrado -- meter isto noutro sitio??).

\chapter{Corpo do Trabalho??}

\chapter{Conclusões}

\chapter{Bibliografia??}

Meter o latex a fazer isto por nós

\appendix

\chapter{Código??}

Será que vai num ficheiro separadamente?

\end{document}
