\documentclass[11pt, language=portuguese]{report}

% encoding
% ------------------------------
\usepackage[T1]{fontenc}
\usepackage[utf8]{inputenc}

% Pacotes para Portugues
% ------------------------------
\usepackage[portuguese]{babel}

% Regra para hifens
% ------------------------------
\usepackage{hyphenat}
\hyphenation{mate-mática recu-perar}

% Outros pacotes (utilitarios)
% ------------------------------
\usepackage{hyperref}

\usepackage{float}

\usepackage{graphicx}
\graphicspath{ {./Imagens/} }

\usepackage{algorithm}
\usepackage{algpseudocode}

\usepackage{ragged2e}

\usepackage{amsthm}

% Definições 
% -----------------------------
\theoremstyle{definition}
\newtheorem{definition}{Definição}


\begin{document}

\title{\includegraphics[width=0.5\textwidth]{minho.png}~\\[1cm] O Sistema Merkle-Hellman Knapsack}

\author{Bernardo Rodrigues\\ \texttt{a79008@alunos.uminho.pt}\\ \and César Silva\\ \texttt{a77518@alunos.uminho.pt}\\ \and Maria Francisca Fernandes\\ \texttt{a72450@alunos.uminho.pt}\\}

\date{Universidade do Minho --- \today}

\maketitle

\begin{abstract}

	Este documento apresenta os vários passos e considerações feitas para implementação do sistema em questão. Assim como, alguns factos relativos a este.  


\end{abstract}

\tableofcontents

\listofalgorithms

\chapter{Introdução}

	Ao contrario do RSA nao da para fazer assinaturas criptograficas - wiki
1976 Diffie Hellman introduzem a ideia de criptografia de chave publica
Este trabalho foi desenvolvido no ambito da Unidade Curricular de \textit{Teoria de Números Computacional}. De entre as escolhas possiveis, foi escolhido estudar o sistema \textit{Merkle-Hellman Knapsack}. \\
Este foi um dos pioneiros da criptografia de chave pública, inventado por \textbf{Ralph Merkle} e por \textbf{Martin Hellman} em 1978.
A ideia por detrás deste sistema é mais simples do que a de sistemas como o \textit{RSA}, assentando no problema  (tendo já sido quebrado -- meter isto noutro sitio??).

\chapter{Implementação}

Ao longo das secções deste capítulo apresentamos os vários algoritmos seguidos para a codificação do sistema.
A implementação destes segue em anexo, assim como ficheiros de texto que acompanham o relatório.

\section{Geração de Permutações}
\label{perm}

Um dos passos da Geração da Chave Pública - \ref{gkey} - consiste em gerar uma permutação de uma sequência. Para tal utilizamos o algoritmo proposto por \textbf{Sandra Sattolo}\autocite{sattolo}.\\
Este itera uma lista - $seq$ - a partir do último índice desta - $n$. Em cada passo calculamos um índice aleatório - $j$ - tal que $1 \le j < n$ e de seguida trocamos os valores de $seq_j$ e $seq_n$. Finalmente, terminámos quando $n = 1$. No nossa implementação usamos listas para guardar as permutações.\\
A implementacão deste pode ser vista em: 

\section{Geração de um sequência super crescente aleatória}
\label{seq}

Um dos componentes da Chave Privada é uma sequência, esta é considerada super crescente se:\\
\begin{definition}
	Consideremos uma sequência de números ${b_1, ..., b_n}$. Esta diz-se super crescente se:
	\begin{center}
		$b_i > \sum_{j = 1}^{i - 1} b_j$ para cada $i$, $2 \le i \le n$. 
	\end{center}
\end{definition}

Como tal, conseguimos deduzir:

\begin{center}
	$b_1 + b_2 + ... + b_k < 2 \times b_k$
\end{center}

Ou seja, precisamos apenas de considerar o último valor gerado para calcular um possivel próximo.
Com isto apresentamos o nosso algoritmo.
\begin{algorithm}[H]
	\caption{Geração da sequência super crescente aleatória}
	\textbf{Recebe}: $n$ - o tamanho da sequência\\
	\textbf{Devolve}: $\{x_1, ..., x_n\}$ - uma sequência super crescente aleatória
	\begin{algorithmic}[1]
		\State $k$ um limite superior aleatóriamente grande
		\State $f$ uma funcão que satisfaz $f(x) > 2 * x$
		\State $x_1 \gets j$ tal que $1 \le j \le k$ aleatório
		\For{$x_i$ com $i:=2$ \textbf{até} $n$ }
			\State $x_i = f(x_{i-1})$
		\EndFor
	\end{algorithmic}
\end{algorithm}

A implementacão deste algoritmo segue em anexo - ()().

\section{Geração de Coprimos}
\label{coprimos}

Também durante a Geração das Chaves - \ref{gkey} - do sistema temos de considerar dois valores, $M$ e $W$.
Dada um sequência super crescente $\{b_1, ..., b_n\}$, $M$ verifica:
NOTA :: ISTO TA UM BOCADO OFF TOPIC, DEPOIS VER
\begin{center}
	$M > \sum_{j = 1}^{N} b_j$
\end{center}

$W$ por sua vez, verifica:
\begin{center}
        $1 \leq W < M$ e $ \gcd(W,M) = 1 $
\end{center}

O algoritmo considerado adopta uma estratégia brute force. Esta é possível devido ao apresentado aqui \autocite{stackex}. Este considera um valor aleatório para $W$ compreendido no intervalo apresentado acima, testando o $\gcd$ de este com $M$. Se forem coprimos o processo para, caso contrário somamos um a este e repetimos o teste. O algoritmo pode ser descrito como.

\begin{algorithm}[H]
	\caption{Gerador de Coprimos - brute force}
	\textbf{Recebe}: $M$ - um número\\
	\textbf{Devolve}: $W$ - um número coprimo com o argumento, $1 \le W < M$
	\begin{algorithmic}[1]
		\State $W \gets$ número aleatório tal que $1 \le W < M$	
		\While{$\gcd(M, W) \ne 1$}
			\State $W = W + 1$
		\EndWhile
	\end{algorithmic}
\end{algorithm}

A sua implementação segue - ref.

\section{Geração da Chave}
\label{gkey}
Dada uma permutação de uma sequência $\pi$.
\\
Dada uma sequência super crescente ${b_1, ..., b_n}$.
\\
Dados dois inteiros co primos W e M.
\\
A geração da chave pode ser dividida em duas subcategorias.

\subsection{Geração da chave Pública}
\begin{definition}\label{def:4}
	São computados ${a_i}$ tal que:
	\begin{center}
		$a_i = W \times b_{\pi_i} mod M $ para cada $i$ tal que $1 \le i \le n$.
        \end{center}
\end{definition}
A chave publica é dada por $({a_1, a_2, ..., a_n})$

\subsection{Geração da chave Privada}
A chave privada é dada por $(\pi, M, W, {(b_1, b_2, b_3, ..., b_n)})$
\autocite{handbook}

\subsection{Versão Multi-Iterada}
Uma variante do algoritmo base de Merkle-Hellman envolve disfarçar a sequência super crescente, através de sucessivas multiplicações modulares.
\\
Neste caso precisamos de fixar outro inteiro, T, que ditará o numero de iterações para geração da chave.
Mais uma vez um inteiro N é fixo como parâmetro de sistema.
\\

Para $1 \le j \le T$ são feitos os cálculos descritos na definição ja nao e definicao. Para calcular a sequencia atual são calculados $M_j$ e $W_j$, da mesma maneira que eram calculados na geração da chaves no algoritmo básico. O $M_j$ é calculado com base na sequência super crescente anterior (índice $j-1$) sendo então depois calculado $W_j$ como descrito na definição \ref{def:3}, tendo como base uma sequencia ${a_1, ..., a_n}$ dada, sendo esta a chave privada.

\section{Encriptacão}
A encriptação é feita a partir da chave pública $a_1, a_2, ... a_n$. É representada a mensagem $m$ a encriptar em binário como uma string de tamanho N. ($m = m_1 m_2 ... m_n$)
\\
Finalmente é calculado um inteiro c tal que:
    $c = \sum_{i = 1}^{N} a_i \times m_i$.

\section{Decriptação}
Para recuperar a mensagem $m$ a partir de $c$, o dono da chave privada deve:
Calcular $d = W^{-1}\times c$ mod $M$.
\\
Encontrar inteiros ${r_1, ..., r_n}, r_i \in \{0,1\}$ tal que $d = r_1 \times b_1 + r_2 \times b_2 + ... + r_n \times b_n$.
\\
Os bits da mensagem são então $m_i = r_{\pi_i}, i = 1, 2, ... n$.
\subsection{Versão Multi-Iterada}
A decriptação desta variante do algoritmo é análoga à sua encriptação, ou seja, os cálculos efetuados são os mesmos do que na versão básica do algoritmo, mas são efetuados mais vezes.
\\
A decriptação é feita a partir de sucessivamente calcularmos:
\begin{center}
$d_j = W^{-1}\times d_{j+1}$ mod $M_j$, para $j = {t, t-1, ..., 1}$
\end{center}
onde $d_{t+1} = c$.
\\
Finalmente só temos de encontrar inteiros ${r_1, ..., r_n}, r_i \in \{0,1\}$ tal que:
\begin{center}
    $d_1 = r_1\times a_1 + r_2\times a_2 + ... + r_n\times a_n$.
\end{center}
E os bits da mensagem são recuperados a partir de aplicarmos a permutação $\pi$.

\chapter{Conclusões}

Após a realizaçao do trabalho e depois de compreendido o problema em questão foi possivel concluir que o sistema Merkle-Hellman Knapsack tal como praticamente todos os cripto-sistemas baseados no \textit{knapsack problem} nao é seguro.Tal como ficou provado, apenas alguns anos após a sua publicaçao, por \textbf{Adi Shamir}. 
Existem várias maneiras de o atacar.È possivel faze-lo,por exemplo, atacando a sua versão Multi-Iterada ou através da divisão em várias partes (falta-me aqui a palavra apropriada) , existindo ainda mais alguns ataques possiveis. Sendo uma demonstraçao de um ataque a este sistema é uma boa aposta para um trabalho futuro.
Para finalizar, resta só concluir que existem poucos sistemas que sejam eguros e poderosos que ainda não tenham sido quebrados. Um deles é o RSA fazendo por isso com que seja um dos, se nao mesmo o mais utilizado. 

\printbibliography

\appendix

\chapter{Código??}

Será que vai num ficheiro separadamente?

\end{document}
